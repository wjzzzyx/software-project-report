\chapter{软件质量特性}

\section{适应性}

本软件系统的客户端适用于任何支持浏览器的软硬件平台,具有良好的适应性。本软件的前端界面开发针对的是IE浏览器,但对其它浏览器也兼容,只是显示效果可能有些不同。

本软件系统的后端(包括题库管理以及试题成型系统)仅支持Windows Server操作系统以及Oracle Database 12c 企业版或更高版本。

\section{可用性}
A. 对于用户注册与管理系统,该软件系统的交互是基于浏览器的,无需安装,即点即用。完全没有使用经验的用户也应能够正常使用该系统;

B. 对于试题管理系统,该软件的交互基于所提供的显示界面。用户必须在服务器上运行管理系统才能够正常进行操作,但也可远程登录到服务器进行相关操作。用户应不需要经过特殊的培训就可以正常使用本系统;

C. 对于考试系统,该软件系统的交互是基于浏览器的,无需安装,即点即用。完全没有使用经验的用户也应能够正常使用该系统。

\section{灵活性}
A. 对于用户注册与管理系统,该软件系统可以在任何支持浏览器的操作系统上运行,包括但不限于windows 7/8/8.1/10, Android以及IOS等。

B. 对于考试系统,该系统能够在任何支持浏览器且显示屏幕分辨率大于1024*768的操作环境下使用。操作环境必须提供鼠标和键盘作为输入工具,且提供摄像头用于拍摄考生照片。

\section{易使用性}
对于各个系统,设计的基本要求是使得用户不需要经过培训即可顺利使用,甚至让用户在完全没有任何使用经验的情况下也可以快速了解系统的大致使用方式。因此,系统各种操作提示清晰、含义明确,可方便使用者进行操作;重要信息显示在明显的地方,使使用者可清楚地了解到信息;信息录入尽量使用选择框,同时应避免填写大量信息而造成不一致。

同时,系统在特定的位置加入了有关的操作说明,以及相关的用户引导,以方便用户了解系统的使用方法。

\section{正确性}
本软件系统的正确性主要在于两方面:程序逻辑的正确性和数据库的正确性。

其中,程序逻辑的正确性通过一系列、多层次的检测来保证,相关的单元测试由ETS提供;数据库的正确性通过数据库管理软件的权限管理、完整性检验等来实现。

考虑到本软件系统的应用场景,其正确性需得到严格保证,不能出现任何错误。

\section{可维护性}
为了维持本软件系统的正常运行,服务器端需定期维护,包括定期每月将数据迁移至永久性存储介质进行备份等。同时,对于题库管理系统,系统内题目的更新应该保持一定的频率,如每两年题库内的题目应该完全更新一次;对于试题成型系统,每次考试开始之前需要由管理人员生成相关的题目。有关于维护所用的接口都应该以便于用户使用的界面提供。

\section{可移植性}
对于试题管理系统,本软件系统可移植到安装了Windows 2012 Server操作系统和Oracle数据库的服务器上。并且对于Windows Server和Oracle的更新版本,在下层接口不改变的情况下该系统也必须保持可移植性。

\section{可靠性}
本软件系统在提供服务期间,应保证随时可靠,不宕机。试题管理系统在考试期间不可宕机;其余情况下,后台系统的每年的宕机时间之和不应超过6小时。

\section{可扩展性}
本软件系统的子功能以模块方式实现,其中用户注册与管理系统、试题管理系统和考试系统分别为独立的模块,它们之间的接口应随时保持统一,不受系统内部各模块修改的影响。系统需要扩展的时候,可以方便地往系统中增加更多模块来完成不同的功能。