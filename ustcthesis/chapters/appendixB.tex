\chapter{需求建模 }
\section{数据流图}
\subsection{顶层数据流图}
<Draw the Top-level DFD here>

在这里画出顶层数据流图

\subsection{层数据流图}
<Draw the Level-0 DFD here>

在这里画出0层数据流图

\subsection{层数据流图}
<Draw the Level-1 DFD here>

在这里画出1层数据流图

\section{数据字典}
\subsection{数据流说明}
\subsubsection{数据流1名称}
<Title of  the data flow should accord with the one in data flow diagram, and the Data description notions should be used.  >

与数据流图中的名称一致,采用数据描述符号说明数据流的内容

\subsubsection{数据流2名称}
<Title of  the data flow should accord with the one in data flow diagram, and the Data description notions should be used   >

与数据流图中的名称一致,采用数据描述符号说明数据流的内容

\subsection{数据存储说明}
\subsubsection{数据存储1名称}
<Title of  the data flow should accord with the one in data flow diagram, and the Data description notions should be used. The arrangement of the data in data store should also be described.>

与数据流图中的名称一致,采用数据描述符号说明数据流的内容,另外还需描述数据排列方式

\subsubsection{数据存储2名称}
<Title of  the data flow should accord with the one in data flow diagram, and the Data description notions should be used.The arrangement of the data in data store should also be described.>

与数据流图中的名称一致,采用数据描述符号说明数据流的内容,另外还需描述数据排列方式

\subsection{加工说明}
\subsubsection{ETS考试与管理系统}
ETS考试与管理系统负责接受考生和ETS管理人员的请求,处理各个请求并且将请求反馈的信息传回给用户和ETS管理人员。

\subsubsection{考生信息管理}
考生信息管理系统负责接受考生的各类请求,例如缴费,更改密码,查阅考试成绩等。系统将会根据各类请求更新数据库,并且将对应的反馈信息传回给考生,例如考生的成绩以及考生是否缴费成功的信息等。

\subsubsection{ETS管理平台}
ETS管理平台负责接受ETS管理人员的各类请求,例如查看某一考生的具体信息,将考生成绩录入到系统中等。平台将根据各类请求向数据库发送查询或者修改请求,并且将对应的请求结果传回给ETS管理人员。

\subsubsection{请求分类}
负责将用户的各类输入请求进行分类以区分用户的具体要求,例如更改信息,查看成绩等。经过具体分类的请求将会传给下一层的不同加工体。

\subsubsection{考试场次查询}
该加工体负责接收考试场次查询的指令,向对应的数据库请求考试场次的信息,并且将对应的信息返回给用户。

\subsubsection{注册}
该加工体负责接收用户请求注册的信息和用户对应的注册信息,将注册信息写入考生信息管理数据库,并将操作成功与否的结果反馈给考生。

\subsubsection{个人信息管理}
该加工体负责接收用户请求查看用户资料的指令,并且向数据库发送查询请求来获得考生信息,并且将具体的信息重新反馈给考生。

\subsubsection{考试报名}
该加工体负责接收用户请求报名的指令,更改用户在考生信息数据库中是否已经报名考试的信息,并且将对应的信息写入到考场数据库中。

\subsubsection{成绩查询}
该加工体负责接收用户查询指令

