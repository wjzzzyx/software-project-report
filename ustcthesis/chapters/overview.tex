\chapter{总体概述}

\section{软件概述}

\subsection{项目介绍}
ETS考试与管理系统是由ETS投资开发,以实现TOFEL、GRE、GMAT考试网络化、信息化,高效为考生服务的重要工作之一。该系统是一个新的独立的项目。它充分考虑了考生的需求,利用网络的便捷性,提高考生获取信息的能力,方便了考试流程。同时该系统也方便了ETS管理人员进行试题、成绩等数据的管理。本系统可以与其他应用系统交互,极大的增强了交互性和可操作性。

\subsection{产品环境介绍}
本系统基于操作系统、数据库管理软件、浏览器和计算机网络的支持,在此基础上,本产品是独立的。

此外,本系统还与其他第三方系统有交互。本系统将与第三方支付平台有资金信息的交流。

\section{软件功能}
本软件系统可以划分为两个子系统,一个是供考生用户使用的,一个是供ETS管理人员使用的,它们实现的功能不同。

供考生使用的客户端系统依赖于浏览器提供的前端界面显示与交互功能。它为考生用户提供的功能有:

新用户注册

个人信息管理

考试时间、地点查询

考试报名

成绩查询

供ETS管理人员使用的服务器端系统主要依赖的是数据库管理软件。它有自己实现的前端界面,为ETS管理人员提供的功能有:

考试信息发布

题库维护

试题成型

试卷分发及测试

\section{用户特征}
本系统涉及的使用者包括两类,考生用户和ETS管理人员用户。

(1)对考生用户的要求:会通过浏览器上网,能根据提示输入正确的信息

(2)对ETS管理人员用户的要求:具有一定的数据库管理经验

\section{假设和依赖关系}
本系统开发过程中,作如下假设:

1、需求不发生大的变动

2、除了下文提及的约束外,不对本系统的运行资源需求加以限制

本系统的依赖关系如下:

1、客户端依赖于浏览器的界面显示与交互

2、服务器端依赖于操作系统和数据库管理软件的支持

3、本系统还将依赖第三方支付平台提供的服务,来为用户提供缴费功能
