\chapter{简介}

\section{编写目的}
本说明书的编写是为了明确ETS考试与管理系统开发的功能需求和性能需求,以标准的语言和表述方式整理系统需求,以便于开发者和用户对系统的理解和认识。

\section{项目背景}
系统名称:ETS考试与管理系统

项目委托单位:ETS

项目开发单位:××公司

\section{预期的读者}
最终用户:分为两类,一类是考生用户,一类是ETS管理人员用户。对于考生用户,主要使用的系统为考试注册报名及缴费系统;对于ETS管理人员,主要使用的系统为题库维护,试题成型,考场管理以及试卷批改等系统。

\section{定义}
考生用户:使用客户端子系统进行账号注册、考试报名等功能的个人。

ETS管理员:使用服务器端系统进行考试信息及成绩发布的个人。

\section{范围}
本文档包括的内容为:

1、系统的功能需求、性能需求

2、所受约束

3、依赖关系

4、可行性及问题分析

本文档不包括的内容为:

1、系统的具体设计方案

2、用户使用指南
